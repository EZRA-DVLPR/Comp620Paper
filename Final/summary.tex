Overall, these four publications discuss different approaches towards improving AI/ML/DL tasks at the hardware level.
The first paper introduces a series of three different hardware technologies that are under active research: GPUs, FPGAs, and ASIC chips.
These three architectures function very differently.
GPUs rely a large amount of simplistic ALUs for processing.
FPGAs are reprogrammable logical gate arrays which allow for static creation/setting of the chip.
ASIC chips are highly specialized and custom-made chips to function for a particular task.
Despite these three hardware approaches being so different, they are capable of improving performance when working on AI tasks.
The other three papers discuss how to improve GPUs and FPGAs performance and power efficiency using several methods, both software and hardware.

Before, discussing the various methods discussed to improve performance for these different architectures, we will preface by saying that ASIC chips are almost completely disregarded.
This is because they are custom-made to whatever purpose the buyer desires.
As a result, they are improved by trying out new hardware techniques created by the buyer themself, not the manufacturer.
Thus, any advancements made are not discussed in depth in this summary of the four previously discussed papers.

With that being said, there do exist methods to improve the three interested technologies.
GPUs can be improved by utilizing ML techniques, improving hardware connections to network switches for data retrieval/submittal, and using advanced network technologies such as Docker and Kubernetes for large scale deployment.
FPGAs can be improved by utilizing ML models to discover bugs while the ML process is running.
By using ML, the developers can discover where to improve the architecture layout of the logical gates.
A different methodology is to utilize a special type of NN in order to take advantage of the hardware improvements made by the FPGA.
ASIC chips instead can be improved by using a novel hardware-software integration for optimal connectivity between the different processing components.

We think that that these technologies are highly desirable espeically given recent advancements in AI with Generative AI models such as ChatGPT, Llama, and Devin.
These various improvements prove capable and utilize both software and hardware improvements.
By combining these approaches, further advancements towards processing for ML tasks may be discovered.

Overall, these improvements prove extremely useful, espeically given the amount of hype around AI as of recent.
There may be additional improvements to improve power efficiency, processing speed, and more for these different hardware architectures.
By understanding ML on a hardware level, as well as on a software level, developers may be able to better deliver on public interests.